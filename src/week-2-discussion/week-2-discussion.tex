
% Document settings
\documentclass[11pt]{article}
\usepackage[margin=1in]{geometry}
\usepackage{graphicx}
\usepackage{wrapfig}
\usepackage{multirow}
\usepackage{setspace}
\usepackage{hyperref}
\usepackage{xurl}
\usepackage{booktabs}
\pagestyle{plain}
\setlength\parindent{0pt}

\begin{document}

\Huge INF632 Homework 1
\normalsize ~

\noindent\rule{\textwidth}{1pt}\vspace{.5em}

\LARGE Compare and Contrast Commodity Devices
\normalsize ~

In this week's reading, you learned about haptics, both the physiological aspects of how we sense and a bit about virtual rendering touch sensations.  In this discussion exercise, I want you to use what your readings  and your own experiences to discussing real world and computer environment analog experiences.

I'll start you off with an example.

\begin{enumerate}
\item Learning and Education
	\begin{itemize}
		\item Real World: Traditional classrooms involve physical interaction with books, writing materials, and hands-on activities.
		\item Computer Environment: Online courses often rely on video lectures, e-books, and interactive quizzes.
		\item Potential Haptic Changes (Enhancements): Haptic feedback could provide tactile hints for answers, simulate page turns in digital textbooks, or enhance engagement through virtual lab activities.  Your Physics labs were probably hands on and allowed you to see and feel Newtonian mechanics.  But you probably haven't had a lab on celestial dynamics, or on molecular bonding, that you were able to explore hands on.
	\end{itemize}

\item Art and Crafting
	\begin{itemize}
		\item Real World: Creating art involves tactile sensations from handling materials like paint, clay, or fabric.
		\item Computer Environment: Digital art software allows users to create art using styluses on tablets or touchscreens.
		\item Potential Haptic Changes: % Haptic feedback could simulate the feeling of different brush strokes or textures, making digital art creation more immersive and intuitive.
	\end{itemize}
	\vspace{2in}

% 3. **Navigation**
%    - *Real World*: Navigating a new area typically involves using maps or asking for directions, often relying on visual and auditory cues.
%    - *Computer Environment*: Digital maps and navigation apps provide visual directions and auditory guidance.
%    - *Potential Haptic Changes*: Adding navigational haptic feedback via smartwatches or phones could offer subtle vibrations for turns or alerts, enhancing situational awareness.

\item Email and Messaging
	\begin{itemize}
		\item Real World: Writing letters involves physically handling paper and the writing instrument, with tactile feedback from pen on paper.
		\item Computer Environment: Composing emails and messages through keyboards and touchscreens.
		\item Potential Haptic Changes: % Haptic keyboards could provide feedback for key presses, mimicking the feeling of traditional writing, and enhancing engagement while typing.
	\end{itemize}
	\vspace{2in}


\item Exercise, Sports, Physical Therapy
	\begin{itemize}
		\item Real World: Participating in sports requires physical activity, coordination, and teamwork. Physical therapy is similar and adds one-on-one guidance such as taps where one needs to be focusing or nudges to move differently.
		\item Computer Environment: Fitness apps and virtual sports games provide guided workouts through screens.
		\item Potential Haptic Changes: %Wearable devices could provide feedback on body position or movement that simulates the experience of engaging with a coach or team during workouts.
	\end{itemize}
	\vspace{2in}

\item What else?
	\begin{itemize}
		\item Real World:
		\vspace{0.5in}
		\item Computer Environment:
		\vspace{0.5in}
		\item Potential Haptic Changes:
	\end{itemize}

\end{enumerate}
% Traveling and Exploration**
%    - *Real World*: Traveling involves physical exploration of new environments, cultures, and experiences.
%    - *Computer Environment*: Virtual tours or simulation games allow users to explore destinations from home.
%    - *Potential Haptic Changes*: Incorporating haptic sensations that simulate terrains, weather, or even local ambiance can deepen the immersion in virtual exploration.

\newpage

\section*{Further Reflection:}

What elements of your daily digital interactions feel less engaging due to the lack of tactile feedback?
\vspace{2in}
How might incorporating haptic feedback change the way you perform these activities?
\vspace{2in}
What specific aspects of the real-world experiences do you think could benefit most from haptic technology?

\end{document}